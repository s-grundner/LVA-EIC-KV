% !TeX program = xelatex
% !TeX encoding = UTF-8
% !TeX spellcheck = en_US
% !BIB program = biber
%% 
%% The above lines help editors like TeXstudio to automatically choose the right tools
%% to compile your LaTeX source file. If your tool does not support these magic comments,
%% you will need to make appropriate manual choices.
%% 
%% You can safely use "pdflatex" instead of "xelatex" if you prefer the pdfLaTeX toolchain.
%% However, pdfLaTeX will not be able to deliver the professional font experience that you
%% will get with XeLaTeX. You can also safely use "lualatex" instead of "xelatex" while
%% preserving the professional font experience if you prefer the LuaLaTeX toolchain.
%% 
%% _Important_: These magic comments should be on the first lines of your source file.
%% 
%%%%%%%%%%%%%%%%%%%%%%%%%%%%%%%%%%%%%%%%%%%%%%%%%%%%%%%%%%%%%%%%%%%%%%%%%%%%%%%%

%%%%%%%%%%%%%%%%%%%%%%%%%%%%%%%%%%%%%%%%%%%%%%%%%%%%%%%%%%%%%%%%%%%%%%%%%%%%%%%%
%% 
%%            JJJJ   K                         K   UUUU         UUUU  
%%            JJJJ   KKKK                   KKKK   UUUU         UUUU  
%%            JJJJ   KKKKKK               KKKKKK   UUUU         UUUU  
%%            JJJJ      KKKKKK         KKKKKK      UUUU         UUUU  
%%            JJJJ         KKKKKK   KKKKKK         UUUU         UUUU  
%%            JJJJ            KKKKKKKKK            UUUU         UUUU  
%%    JJ     JJJJJ               KKK               UUUUU       UUUUU  
%%  JJJJJJJJJJJJJ    KKKKKKKKKKKKKKKKKKKKKKKKKKK    UUUUUUUUUUUUUUU   
%%    JJJJJJJJJ      KKKKKKKKKKKKKKKKKKKKKKKKKKK      UUUUUUUUUUU     
%% 
%% This is an example file for using the JKU LaTeX technical report template
%% for your technical report.
%% 
%% Template created by Michael Roland (2021)
%% 
%%%%%%%%%%%%%%%%%%%%%%%%%%%%%%%%%%%%%%%%%%%%%%%%%%%%%%%%%%%%%%%%%%%%%%%%%%%%%%%%

%%%%%%%%%%%%%%%%%%%%%%%%%%%%%%%%%%%%%%%%%%%%%%%%%%%%%%%%%%%%%%%%%%%%%%%%%%%%%%%%
%% 
%% Document class: This is a koma-script article.
%% 
\documentclass[a4paper,oneside,11pt,english]{scrartcl}
%% 
%% The comma-separated list in square brackets are class options.
%% Useful options that you might want to use:
%% 
%% Paper size:
%%  * a4paper ... A4 paper size
%% 
%% Optimize for single-sided or double-sided printing:
%%  * oneside ... single-sided
%%  * twoside ... double-sided
%% 
%% Base font size:
%%  * 10pt ... 10-pt font is used for normal text
%%  * 11pt ... 11-pt font is used for normal text
%% 
%% Define document languages (the last specified language becomes the document default
%% language):
%%  * ngerman ... German
%%  * english ... English
%%  * ...
%% 
%% Alternate document classes: The JKU report template supports the koma-script classes
%% `scrartcl', `scrreprt' and `scrbook'. The article class `scrartcl' is well-suited
%% for a typical technical report. However, `scrbook' or `scrreprt' may be better
%% suited for longer reports since they permit structuring your work in chapters.
%%  
%% _Important_: The document class should be the first line of LaTeX code in your main
%% source file. Do not place anything but comments / magic comments above that line (unless
%% you really know what you are doing).
%% 
%%%%%%%%%%%%%%%%%%%%%%%%%%%%%%%%%%%%%%%%%%%%%%%%%%%%%%%%%%%%%%%%%%%%%%%%%%%%%%%%

%%%%%%%%%%%%%%%%%%%%%%%%%%%%%%%%%%%%%%%%%%%%%%%%%%%%%%%%%%%%%%%%%%%%%%%%%%%%%%%%
%% 
%% Treat input files as UTF-8 encoded. Make sure to always load this when you use pdfLaTeX
%% so that pdfLaTeX knows how to read and interpret characters in this source file.
%% 
\usepackage[utf8]{inputenc}
%% 
%%%%%%%%%%%%%%%%%%%%%%%%%%%%%%%%%%%%%%%%%%%%%%%%%%%%%%%%%%%%%%%%%%%%%%%%%%%%%%%%

%%%%%%%%%%%%%%%%%%%%%%%%%%%%%%%%%%%%%%%%%%%%%%%%%%%%%%%%%%%%%%%%%%%%%%%%%%%%%%%%
%% 
%% Use the JKU LaTeX technical report template for this document.
%% 
\usepackage[TNF,equalmargins,techreport,nofancyfonts]{jkureport}
%% 
%% The comma-separated list in square brackets are theme options. Useful options that you
%% might want to use:
%% 
%% Document type:
%%  * phdthesis     ... PhD thesis.
%%  * mathesis      ... Master's thesis.
%%  * diplomathesis ... Diploma thesis.
%%  * bathesis      ... Bachelor's thesis.
%%  * seminarreport ... Seminar report.
%%  * techreport    ... Technical report.
%% 
%% Color scheme selection options:
%%  * JKU  ... Use JKU (gray) color scheme (this is the default if no scheme is selected).
%%  * BUS  ... Use Business School color scheme.
%%  * LIT  ... Use Linz Institute of Technology color scheme.
%%  * MED  ... Use MED faculty color scheme.
%%  * RE   ... Use RE faculty color scheme.
%%  * SOE  ... Use School of Education color scheme.
%%  * SOWI ... Use SOWI faculty color scheme.
%%  * TNF  ... Use TNF faculty color scheme.
%% 
%% Space-efficient monospace font options (requires XeTeX/LuaTeX):
%%  * compactmono   ... Use condensed fixed-width font everywhere.
%%  * nocompactverb ... Do not use condensed fixed-width font for verbatim and listings.
%% 
%% Style-breaking options:
%%  * noimprint      ... Do not insert imprint on title pages.
%%  * nojkulogo      ... Do not insert JKU & K logos on title pages.
%%  * capstitle      ... Set document title in capital letters.
%%  * nofancyfonts   ... Do not use custom TTF fonts with XeTeX/LuaTeX / supress pdfLaTeX warning.
%%  * equalmargins   ... Decrease the outer page margin to have both page margins of equal size
%%                       (the additional outer margin is intentional and to be used for
%%                       anotations; equalmargins also causes the text width to be
%%                       significantly larger than optimal for reading).
%% 
%% Experimental options:
%%  * mathastext ... Use standard document fonts (enhanced with symbols from Fira Math font
%%                   when using XeTeX/LuaTeX) in math mode.
%% 
%% Advanced options:
%%  * noautopdfinfo     ... Do not automatically try to add pdfinfo with hyperref from document
%%                          metadata fields.
%%  * logopath={<path>} ... Set the path where the theme can find its own logo resources. This
%%                          should typically be a relative path and the default is `./logos'.
%%  * fontpath={<path>} ... Set the path where the theme can find its own font resources. This
%%                          should typically be a relative path and the default is `./fonts'.
%% 
%% Hint: Boolean options can be used in the forms `option' or `option=true' the enable the
%% option and `nooption' or `option=false' to disable the option.
%% 
%%%%%%%%%%%%%%%%%%%%%%%%%%%%%%%%%%%%%%%%%%%%%%%%%%%%%%%%%%%%%%%%%%%%%%%%%%%%%%%%

%%%%%%%%%%%%%%%%%%%%%%%%%%%%%%%%%%%%%%%%%%%%%%%%%%%%%%%%%%%%%%%%%%%%%%%%%%%%%%%%
%% 
%% This is the place where you can load additional packages. If you want to load
%% a package `biblatex', you would use the command `\usepackage{biblatex}'.
%% 

%% 
%%%%%%%%%%%%%%%%%%%%%%%%%%%%%%%%%%%%%%%%%%%%%%%%%%%%%%%%%%%%%%%%%%%%%%%%%%%%%%%%

\begin{document}
%%%%%%%%%%%%%%%%%%%%%%%%%%%%%%%%%%%%%%%%%%%%%%%%%%%%%%%%%%%%%%%%%%%%%%%%%%%%%%%%
%% 
%% Report information and title page
%% 

%% Command \title{title}: sets the title of your report
\title{Polyphonic MIDI-Synth}

%% Command \titleshort{short title}: sets an abbreviated version of the report title for page heads
%\titleshort{Optional space for your abbreviated title}

%% Command \subtitle{subtitle}: sets the subtitle for seminar/technical reports (not used for theses)
\subtitle{KV - Integrated Circuit Design\\%
    \usekomafont{subtitlesmall}%
    \hfill\\%
    WiSe25\\
    \hfill\\%
}

%% Command \author{name}: sets the author name(s); separate multiple authors with \and; use \prefix{}
%%   and \suffix{} to add academic titles and suffixes (if needed); use \affiliation{} to add an
%%   affiliation, use \authornewline to add line breaks (e.g. to separate authors from contact
%%   information), use \authormail{}, \authorweb{}, \authorphone{} and \authorfax{} to add contact
%%   information)
\author{%
    Simon Grundner
    \affiliation{Institute for Integrated Circuits}
    \authornewline
    \authormail{k12136610@students.jku.at}
    \authorweb{https://jku.at/iic}
    \authornewline
}

%% Command \date{YYYY-MM-DD}: set the day of publication (defaults to today)
%\date{2020-04-09}

%% Command \partnerlogo{filename}: use filename as partnerlogo, filename may be blank to disable the logo
\partnerlogo{logos/iic}

%% Command \revisionblock{text}: set the document revision block on the title page
%\revisionblock{Space for your revision block, acknowledgements, etc.}

%% Command \reportnumber{number}: set the report number
%\reportnumber{Space for your report number}

%% Command \setbottommark{text}: set the bottom mark (in document footer)
\setbottommark{Simon Grundner, k12136610}

%% Command \abstract{text}: set the document abstract on the title page
%\abstract{Space for your (short) abstract.}

%% Command \keywords{text}: set the document keywords
\keywords{}


%% Finally, print the title page using the above information:
\maketitle
%% 
%%%%%%%%%%%%%%%%%%%%%%%%%%%%%%%%%%%%%%%%%%%%%%%%%%%%%%%%%%%%%%%%%%%%%%%%%%%%%%%%

%%%%%%%%%%%%%%%%%%%%%%%%%%%%%%%%%%%%%%%%%%%%%%%%%%%%%%%%%%%%%%%%%%%%%%%%%%%%%%%%
%% 
%% Add a table of contents
%% 

%% Print the table of contents
\tableofcontents

%% Make sure to start the list of acronyms on a new odd page (odd is only relevant in twoside layout)
%\cleardoubleoddpage
%% Include list of acronyms (optional and often not necessary)
%\import{./}{acronyms}

%% 
%%%%%%%%%%%%%%%%%%%%%%%%%%%%%%%%%%%%%%%%%%%%%%%%%%%%%%%%%%%%%%%%%%%%%%%%%%%%%%%%

%%%%%%%%%%%%%%%%%%%%%%%%%%%%%%%%%%%%%%%%%%%%%%%%%%%%%%%%%%%%%%%%%%%%%%%%%%%%%%%%
%% 
%% Add your report sections ...
%% 
\section{Tiny Tapeout}

\subsection{How it works}

A MIDI device, capable of sending Note ON and Note OFF commands is connected via a MIDI PMOD. The ASIC listens on channel 1 for key presses and synthesizes up to 7 voices. Each square oscillator voice is routed to a different output pin 0-6 and can be mixed together externally. Additionally, a PWM signal is output on the final output pin 7, which encodes the amount of currently playing voices. The PWM Signal may be used to provide a reference voltage (through heavy low pass filtering) to compensate the gain, which arises through mixing (adding) several outputs together.

\subsection{How to test}

Connect a MIDI device. Press a key and Measure output pin 0 with an oscilloscope for the correct frequency. Press multiple keys simultaneously, and check each pin for the expected output.

\subsection{External Hardware}

\begin{itemize}
    \item MIDI Controller (Piano, Pads)
    \item MIDI DIN connector PMOD as a physical layer for the differential midi signal
    \item (Optional) External Mixing circuitry. Example circuit provided in the repository
    \item Speaker (High impedance when used without a driver)
\end{itemize}

\section{Notes}

use keyboard with ps2 interface for midi input.

start by implementing one osc.

define number of voices (number of independent osc)

repeat architecture for each voice 

just route audio to seperate digital outputs and mix together with external analog OpAmp circuit

Use one output pin to output a pwm signal whose duty cycle is inversely proportional to the number of currently playing voices.

This signal can then be lowpass filtered to create an analogue voltage to be used to downscale the summed outputs to retain a constant audio volume.

Inputs:

\begin{itemize}
    \item midi / ps2 shared pins
    \item switch between midi and ps2
\end{itemize}

Outputs:

\begin{itemize}
    \item 0-6: Square soundwave
    \item 7: PWM (DC = 1/N, N ... currently active outputs)
\end{itemize}

%% 
%%%%%%%%%%%%%%%%%%%%%%%%%%%%%%%%%%%%%%%%%%%%%%%%%%%%%%%%%%%%%%%%%%%%%%%%%%%%%%%%

\end{document}
\endinput
